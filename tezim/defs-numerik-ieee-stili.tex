%
\usepackage{color}
\usepackage{times}
\usepackage{amssymb,amsmath,mathptmx,amsbsy,bm}
\usepackage{caption}            %%
%\usepackage{floatflt}
%\usepackage[dvipdfm]{graphicx}
\usepackage{graphics}
\usepackage{wrapfig}
\usepackage{epsfig}
\usepackage{enumerate}
\usepackage{rotating}
\usepackage{multirow}
\usepackage{subfigure}
\usepackage{colortbl}
\usepackage{pstricks}
\usepackage{pst-plot}
%\usepackage{cite}%
\usepackage{latexsym}           % 
%\usepackage{subeqn}
\usepackage{rotating}
%\usepackage{amssymb}
%\usepackage{hyperref}
%\usepackage{url}
\usepackage{fixltx2e} % Bu paketi sembollerde text ler için subscript yazmakta yardımcı olması için ekliyoruz.

\usepackage{textcomp}


%Bold Equation Number, Unbold Reference
%\makeatletter
%\def\tagform@#1{\maketag@@@{\bfseries(\ignorespaces#1\unskip\@@italiccorr)}}
%\renewcommand{\eqref}[1]{\textup{{\normalfont(\ref{#1}}\normalfont)}}
%\makeatother

%\renewcommand{\theequation}{{\bf\arabic{chapter}.\arabic{equation}}}
%%%%\renewcommand{\thefootnote}{\hskip \arabic{footnote}}

\def\be{\begin{equation}} %
\def\ee{\end{equation}}%
\def\beq{\begin{eqnarray}}%
\def\eeq{\end{eqnarray}}%
\def\bse{\begin{subequations}}%
\def\ese{\end{subequations}}%
\def\nonu{\nonumber}%
\def\psibar{\overline{\psi}} %
\def\Delslash{\partial\!\!\!\!\!/}%
\def\Gslash{G\!\!\!\!\!/}%
\def\Lmbdslash{\lambda\!\!\!\!\!/}%
\def\Jslash{J\!\!\!\!\!/}%
\def\pslash{p\!\!\!\!/}%
\def\qslash{q\!\!\!\!/}%
\def\kslash{k\!\!\!\!/}%
\def\[{\left[}
\def\]{\right]}
\def\({\left(}
\def\){\right)}

\def\Lag{{\cal L}}    % 
\def\D{{\cal D}}      % 
\def\H{{\cal H}}      % 
\def\Z{{\cal Z}}      % 
\def\S{{\textbf{S}}}  % 
\def\d{{\rm d}}%
\def\dt{{\rm{dt}}}%
\def\Tr{{\rm{Tr}}}%
\def\gm{\gamma^{\mu}}
\def\ga{\gamma^{\alpha}}
\def\gb{\gamma^{\beta}}
\def\gn{\gamma^{\nu}}
\def\gs{\gamma^{\sigma}}
\def\gl{\gamma^{\lambda}}
\def\gr{\gamma^{\rho}}
\def\gd{\gamma^{\delta}}


\usepackage{csquotes}

%\usepackage[%
%backend=biber,
%style=authoryear-comp,
%bibstyle=apa,
%firstinits=true,
%terseinits=true,
%eprint=false,
%doi=false,
%isbn=false,
%url=false,
%apamaxprtauth=20,
%maxcitenames=2,
%uniquelist=false,
%citestyle=numeric-comp,
%sorting=none,
%natbib]{biblatex}

\usepackage[%
backend=biber,
style=ieee,
eprint=false,
doi=false,
isbn=false,
url=false,
natbib]{biblatex}

%\DeclareFieldFormat{labelnumberwidth}{\mkbibbrackets{#1}}
%
%\defbibenvironment{bibliography}
%{\list
%	{\printtext[labelnumberwidth]{%
%			\printfield{labelprefix}%
%			\printfield{labelnumber}}}
%	{\setlength{\labelwidth}{\labelnumberwidth}%
%		\setlength{\leftmargin}{\labelwidth}%
%		\setlength{\labelsep}{\biblabelsep}%
%		\addtolength{\leftmargin}{\labelsep}%
%		\setlength{\itemsep}{\bibitemsep}%
%		\setlength{\parsep}{\bibparsep}}%
%	\renewcommand*{\makelabel}[1]{\hss##1}}
%{\endlist}
%{\item}



%\DeclareLanguageMapping{american}{american-apa}
%\DefineBibliographyStrings{american}{andothers={ve\ \addabbrvspace ark\adddot}}
%\DefineBibliographyStrings{american}{and={\hspace{-0.1cm}}}
%



%bib itemlar arası mesafe
\setlength\bibitemsep{6pt}

% bir bibitem iki sayfaya bolunmemeli
\patchcmd{\bibsetup}{\interlinepenalty=5000}{\interlinepenalty=10000}{}{}

%% and yada & ifadesini siliyor..
%\DeclareDelimFormat*{finalnamedelim}
%{\ifnum\value{liststop}>2 \finalandcomma\fi\addspace ve\space}
%
%% the bibliography also needs another conditional, so we can't wrap
%% everything up with just the two lines above
%\DeclareDelimFormat[bib,biblist]{finalnamedelim}{%
%	\ifthenelse{\value{listcount}>\maxprtauth}
%	{}
%	{\ifthenelse{\value{liststop}>2}
%		{\finalandcomma\addspace\bibstring{}}
%		{\addspace\bibstring{}}}}
%
%% this is a special delimiter to solve the bugs reported in
%% https://tex.stackexchange.com/q/417648/35864
%\DeclareDelimFormat*{finalnamedelim:apa:family-given}{%
%	\ifthenelse{\value{listcount}>\maxprtauth}
%	{}
%	{\finalandcomma\addspace\bibstring{}}}
%
%% virgul
%
%\DeclareDelimFormat{nameyeardelim}{\space}
%
%
%
%
%\usepackage{xpatch}
%
%\xpatchnameformat{labelname}
%{\ifciteseen}{\ifnumcomp{\value{listtotal}}{>}{\value{maxnames}}}{}{}
%
%
%
%\DefineBibliographyExtras{american}{\def\finalandcomma{\addcomma}}
%
%\xpatchbibmacro{editorinauthpos}{%
%	\clearname{editor}%
%	\setunit{\adddot\addspace}%
%}{%
%	\clearname{editor}%
%	\setunit{\addspace}%
%}{}{}
%
%
%\renewcommand{\multicitedelim}{\addcomma\space}
%
%\renewcommand{\labelnamepunct}{\addspace}
%\renewcommand{\bibpagespunct}{\addcolon\space}
%
%
%\xpatchbibmacro{journal+issuetitle}{%
%	\usebibmacro{journal}%
%	\setunit*{\addcomma\addspace}%
%}
%{%
%	\usebibmacro{journal}%
%	\setunit*{\addspace}%
%}{}{}
%
%% Omit commas in reversed names
%% soyadından sonraki virgülü siliyor.
%\renewcommand*{\revsdnamepunct}{}
%
%\renewbibmacro*{name:apa:last-first}[5]{%
%	\ifuseprefix
%	{\usebibmacro{name:delim}{#4#1}%
%		\usebibmacro{name:hook}{#4#1}%
%		\ifblank{#4}{}{%
%			\mkbibnameprefix{#4}%
%			\ifpunctmark{'}{}{\addhighpenspace}}%
%		\mkbibnamelast{#1\isdot}%
%		\ifblank{#2}{}{\revsdnamepunct\bibnamedelimd\mkbibnamefirst{#3}\isdot%
%			\ifthenelse{\value{uniquename}>1}
%			{\addspace\mkbibbrackets{#2}}
%			{}}%
%		\ifblank{#5}{}{\addcomma\bibnamedelimd\mkbibnameaffix{#5}\isdot}}
%	{\usebibmacro{name:delim}{#1}%
%		\usebibmacro{name:hook}{#1}%
%		\mkbibnamelast{#1}\isdot
%		\ifblank{#2#4}{}{\revsdnamepunct}%
%		\ifblank{#2}{}{\bibnamedelimd\mkbibnamefirst{#3}%
%			\ifthenelse{\value{uniquename}>1}
%			{\addspace\mkbibbrackets{#2}}
%			{}}%
%		\ifblank{#4}{}{%
%			\addhighpenspace\mkbibnameprefix{#4}%
%			\ifpunctmark{'}{}{\addhighpenspace}}%
%		\ifblank{#5}{}{\addcomma\addlowpenspace\mkbibnameaffix{#5}\isdot}}}
%
%
%% ilk isimlerin sonundaki noktaları siliyor
%\makeatletter
%\renewbibmacro*{name:last-first}[4]{%
%	\ifuseprefix
%	{\usebibmacro{name:delim}{#3#1}%
%		\usebibmacro{name:hook}{#3#1}%
%		\ifblank{#3}{}{%
%			\ifcapital
%			{\mkbibnameprefix{\MakeCapital{#3}}\isdot}
%			{\mkbibnameprefix{#3}\isdot}%
%			\ifpunctmark{'}{}{\bibnamedelimc}}%
%		\mkbibnamelast{#1}\isdot
%		\ifblank{#4}{}{\bibnamedelimd\mkbibnameaffix{#4}\isdot}%
%		%      \ifblank{#2}{}{\addcomma\bibnamedelimd\mkbibnamefirst{#2}\isdot}}% DELETED
%		\ifblank{#2}{}{\bibnamedelimd\mkbibnamefirst{#2}\isdot}}% NEW
%	{\usebibmacro{name:delim}{#1}%
%		\usebibmacro{name:hook}{#1}%
%		\mkbibnamelast{#1}\isdot
%		\ifblank{#4}{}{\bibnamedelimd\mkbibnameaffix{#4}\isdot}%
%		%      \ifblank{#2#3}{}{\addcomma}% DELETED
%		\ifblank{#2}{}{\bibnamedelimd\mkbibnamefirst{#2}\isdot}%
%		\ifblank{#3}{}{\bibnamedelimd\mkbibnameprefix{#3}\isdot}}}
%\makeatother
%
%\DeclareNameAlias{sortname}{last-first}
%
%% ilk isimlerin arasına boşluk koyuyor
%%\renewrobustcmd*{\bibinitdelim}{\,}
%
%
%% en sondaki yazarın sonuna nokta koyuyordu o silindi.
%\renewbibmacro*{author}{%
%	\ifnameundef{author}
%	{\usebibmacro{labeltitle}}
%	{\printnames[apaauthor][-\value{listtotal}]{author}%
%		\setunit*{\addspace}%
%		\printfield{nameaddon}%
%		\ifnameundef{with}
%		{}
%		{\setunit{}\addspace\mkbibparens{\printtext{\bibstring{with}\addspace}%
%				\printnames[apaauthor][-\value{listtotal}]{with}}
%			\setunit*{\addspace}}}%
%	\setunit{\addspace}\newblock%
%	\usebibmacro{labelyear+extradate}}
%
%\renewbibmacro*{editorinauthpos}{%
%	\global\booltrue{bbx:editorinauthpos}%
%	\printnames[apaauthor][-\value{listtotal}]{editor}%
%	\setunit{\addspace}%
%	\ifnameundef{editor}
%	{}
%	{\printtext[parens]{\usebibmacro{apaeditorstrg}{editor}}%
%		% need to clear editor so we don't get an "In" clause later
%		% But we also need to set a flag to say we did this so we
%		% don't lose sight of the fact we once had an editor for
%		% various year placement tests
%		\clearname{editor}%
%		\setunit{\addspace}%
%		\usebibmacro{labelyear+extradate}%
%		\setunit{\adddot\addspace}}}
%
%% yuzde ısareti için
%\DeclareSourcemap{
%	\maps[datatype = bibtex]{
%		\map{
%			\step[fieldsource = abstract,
%			match = \regexp{([^\\])\%},
%			replace = \regexp{$1\\\%}]
%		}
%	}
%}

%%%%%%%%%%%%%%%%%%%%%%%%%%%%%%%%%%%%%%%%%%%%%%%%%%%%%%%%%%%%%%%%%%%%%%%%%%%%%%%%%%%%%%%%%%%%%%%%%%%%%%%%%%%%%%%%%%%%%%%%%%

\usepackage{tabularx}
\newcolumntype{L}[1]{>{\raggedright\arraybackslash}p{#1}}
\newcolumntype{C}[1]{>{\centering\arraybackslash}p{#1}}
\newcolumntype{R}[1]{>{\raggedleft\arraybackslash}p{#1}}


%\newtheorem{thm}{Teorem} % is it right?
%\newtheorem{ex}{Voorbeeld}

%% footnote ikinci satır için.. daha sonra bakılacak
%\makeatletter
%\renewcommand\@makefntext[1]{\leftskip=2em\hskip0em\@makefnmark#1}
%\makeatother